\chapter{Zaključak i budući rad}
				
		     Zadatak naše grupe bio je razvoj web aplikacije za konferencije. Omogućeno je prećenje dolaska sudionika na konferenciju, sudjelovanja sudionika u događanjima na konferenciji i davanje svih potrebnih informacija sudionicima. Rad na izradi aplikaciji trajao je 16 tjedana. Provedba projekta je bila kroz dvije faze.
   
             Prva faza projekta uključivala je okupljanje tima za razvoj aplikacije, dodjelu projektnog zadatka i dokumentiranje zahtjeva. Kvalitetna provedba prve faze uvelike je olakšala daljnji rad pri realizaciji osmišljenog sustava. Izrađeni obrasci i dijagrami (obrasci uporabe, sekvencijski dijagrami, model baze podataka, dijagram razreda) bili su od pomoći podtimovima zaduženima za razvoj \textit{backenda} i \textit{frontenda}. 
		
		     Dok se u prvoj fazi najvećim dijelom fokusiralo na grupnom pronalaženju optimalnog rješenja, druga faza projekta bila je puno intenzivnija po pitanju samostalnog rada članova. Članovi tima bili su bez iskustva rada na sličnim projektima te su dosta vremena potrošili na upoznavanje i učenje odabranih alata za izradu aplikacije. Također puno vremena potrošeno je na puštanje aplikacije u pogon. Osim realizacije rješenja i \textit{deploy}-a, u drugoj fazi je bilo potrebno dokumentirati ostale UML dijagrame i izraditi popratnu dokumentaciju kako bi budući korisnici mogli lakše koristiti. Dokumentiranje iz prve faze uštedjelo nam je na vremenu tijekom konkretne izrade aplikacije, iako je bilo manjih izmjena i dorada kako se aplikacija razvijala.

             Komunikacija između članova tima bila je većinom putem WhatsAppa, a manjim dijelom i Discorda. Moguće proširenje projekta bila bi izrada mobile aplikacije za lakše i efikasnije korištenje stalnim korisnicima.

             Rad na ovom projektu bilo je jako vrijedno te ujedno i novo iskustvo članovima tima. Naučili smo puno o radu u timu te smo proširili znanje u smislu poznavanja novih alata. Svjesni smo da aplikacija ima puno prostora za poboljšanje, no iznimno smo zadovoljni postignutim rezultatom projekta upravo zbog neiskustva svih članova tima.
		
		\eject 